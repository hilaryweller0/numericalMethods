%% LyX 2.2.4 created this file.  For more info, see http://www.lyx.org/.
%% Do not edit unless you really know what you are doing.
\documentclass[12pt,english]{article}
\usepackage{mathptmx}
\usepackage[T1]{fontenc}
\usepackage[latin9]{inputenc}
\usepackage{geometry}
\geometry{verbose,tmargin=2.5cm,bmargin=2.5cm,lmargin=2.5cm,rmargin=2.5cm}
\usepackage{color}
\usepackage{url}
\usepackage{amstext}
\usepackage{setspace}
\usepackage[authoryear]{natbib}
\usepackage{babel}
\begin{document}
\begin{doublespace}
\noindent \begin{center}
{\Large{}Implicit Advection}{\Large\par}
\par\end{center}
\end{doublespace}

The one dimensional linear advection equation in flux form is
\begin{equation}
\frac{\partial\phi}{\partial t}+\frac{\partial u\phi}{\partial x}=0.
\end{equation}
The generic finite volume method for solving this equation is
\begin{equation}
\frac{\phi_{j}^{n+1}-\phi_{j}^{n}}{\Delta t}=-\frac{u_{j+\frac{1}{2}}\phi_{j+\frac{1}{2}}-u_{j-\frac{1}{2}}\phi_{j-\frac{1}{2}}}{\Delta x}.\label{eq:FV}
\end{equation}
Assuming that $u>0$, the linear upwind approximation for $\phi_{j+\frac{1}{2}}$
that is implemented in OpenFOAM is
\begin{equation}
\phi_{j+\frac{1}{2}}=\phi_{j}+\frac{\phi_{j+1}-\phi_{j-1}}{4}.\label{eq:LUinterp}
\end{equation}
This is a correction on first order upwind. $\phi_{j}$ is the upwind
approximation and $\frac{\phi_{j+1}-\phi_{j-1}}{4}$ is the correction
to make it second order accurate. 

This can be combined with various time stepping schemes. To describe
the time stepping schemes I will use the notation:
\begin{equation}
F=-\frac{u_{j+\frac{1}{2}}\phi_{j+\frac{1}{2}}-u_{j-\frac{1}{2}}\phi_{j-\frac{1}{2}}}{\Delta x}
\end{equation}
so that the generic finite volume method is $\frac{\phi_{j}^{n+1}-\phi_{j}^{n}}{\Delta t}=F$.
It is also useful to have short-hands for linear upwind:
\begin{equation}
F_{\text{LU}}=-\frac{u_{j+\frac{1}{2}}\left(\phi_{j}+\frac{\phi_{j+1}-\phi_{j-1}}{4}\right)-u_{j-\frac{1}{2}}\left(\phi_{j-1}+\frac{\phi_{j}-\phi_{j-2}}{4}\right)}{\Delta x}\label{eq:LUflux}
\end{equation}
and first order upwind:
\begin{equation}
F_{\text{U}}=-\frac{u_{j+\frac{1}{2}}\phi_{j}-u_{j-\frac{1}{2}}\phi_{j-1}}{\Delta x}
\end{equation}
so that $F_{\text{LU}}=F_{\text{U}}+F_{\text{LUC}}$ where $F_{\text{LUC}}$
is the correction on first order upwind:
\begin{equation}
F_{\text{LUC}}=-\frac{u_{j+\frac{1}{2}}\left(\frac{\phi_{j+1}-\phi_{j-1}}{4}\right)-u_{j-\frac{1}{2}}\left(\frac{\phi_{j}-\phi_{j-2}}{4}\right)}{\Delta x}.
\end{equation}
Linear upwind can also be written as a finite difference scheme by
substituting (\ref{eq:LUinterp}) into (\ref{eq:FV}):
\begin{equation}
\frac{\phi_{j}^{n+1}-\phi_{j}^{n}}{\Delta t}=-u\frac{\phi_{j+1}+3\phi_{j}-5\phi_{j-1}+\phi_{j-2}}{4\Delta x}.\label{eq:LU_FD}
\end{equation}
This is different from the more compact version of linear upwind reported
in \url{https://en.wikipedia.org/wiki/Upwind_scheme} but you can
check that they are both second order by substituting in $\phi=x^{2}$,
$\Delta x=1$ and check that they both calculate the correct gradient
of $\phi$. Either or both scheme can be analysed. But they need to
be combined with a time stepping scheme ...

The second order, explicit Heun time stepping scheme can then be written
\begin{eqnarray*}
\phi_{j}^{\prime} & = & \phi_{j}^{n}+\Delta tF^{n}\\
\phi_{j}^{n+1} & = & \phi_{j}^{n}+\frac{\Delta t}{2}\left(F^{n}+F^{\prime}\right)
\end{eqnarray*}
and Crank-Nicolson is:
\begin{equation}
\phi_{j}^{n+1}=\phi_{j}^{n}+\frac{\Delta t}{2}\left(F^{n}+F^{n+1}\right)
\end{equation}
or with off centering, the explicit scheme is
\begin{eqnarray*}
\phi_{j}^{\prime} & = & \phi_{j}^{n}+\Delta tF^{n}\\
\phi_{j}^{n+1} & = & \phi_{j}^{n}+\Delta t\left((1-\alpha)F^{n}+\alpha F^{\prime}\right)
\end{eqnarray*}
and the implicit scheme is
\begin{equation}
\phi_{j}^{n+1}=\phi_{j}^{n}+\Delta t\left((1-\alpha)F^{n}+\alpha F^{n+1}\right).\label{eq:CN_alpha}
\end{equation}
If we work out the matrix equation that needs to be solved to use
(\ref{eq:CN_alpha}) with $F=F_{\text{LU}}$, we find that the matrix
is only diagonally dominant for a sufficiently small Courant number.

\textcolor{red}{James - what is this Courant number?}

However if we use first order upwind, $F=F_{\text{U}}$, the matrix
is always diagonally dominant (\textcolor{red}{check}). Therefore
we need to use linear upwind as an explicit correction on first order
upwind:
\begin{eqnarray*}
\phi_{j}^{\prime} & = & \phi_{j}^{n}+\Delta t\left((1-\alpha)F^{n}+\alpha F_{\text{U}}^{\prime}+\alpha F_{\text{LUC}}^{n}\right)\\
\phi_{j}^{n+1} & = & \phi_{j}^{n}+\Delta t\left((1-\alpha)F^{n}+\alpha F_{\text{U}}^{n+1}+\alpha F_{\text{LUC}}^{\prime}\right).
\end{eqnarray*}
But is this still stable for large Courant numbers? Do we need more
iterations, eg:
\begin{eqnarray*}
\phi_{j}^{\prime} & = & \phi_{j}^{n}+\Delta t\left((1-\alpha)F^{n}+\alpha F_{\text{U}}^{\prime}+\alpha F_{\text{LUC}}^{n}\right)\\
\phi_{j}^{\prime\prime} & = & \phi_{j}^{n}+\Delta t\left((1-\alpha)F^{n}+\alpha F_{\text{U}}^{\prime\prime}+\alpha F_{\text{LUC}}^{\prime}\right)\\
\phi_{j}^{\prime\prime\prime} & = & \phi_{j}^{n}+\Delta t\left((1-\alpha)F^{n}+\alpha F_{\text{U}}^{\prime\prime\prime}+\alpha F_{\text{LUC}}^{\prime\prime}\right)\\
\phi_{j}^{n+1} & = & \phi_{j}^{n}+\Delta t\left((1-\alpha)F^{n}+\alpha F_{\text{U}}^{n+1}+\alpha F_{\text{LUC}}^{\prime\prime\prime}\right).
\end{eqnarray*}
This can be written more generally as a Runge-Kutta IMEX scheme \citep[see eg][]{WLW13}.

These schemes with multiple stages can be difficult to analyse. You
need to calculate intermediate amplification factors \citep[see eqns 13,14 of][]{WLW13}.
The resulting amplification factors are complicated, complex and not
obvious for which conditions the schemes are stable. The easiest thing
to do is to conduct the analysis numerically: set a range of values
of $k\Delta x$, a range of values of $c$ and with a script, calculate
and plot an array of values of $|A|$. 

\bibliographystyle{abbrvnat}
\bibliography{numerics}

\end{document}
